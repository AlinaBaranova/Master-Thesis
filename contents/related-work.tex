\chapter{Related Work}
\label{sec:related-work}

This section is devoted to the overview of the existing systems that offer the user a possibility to learn languages through song lyrics. Only websites and applications that involve exercises based on songs texts are examined. Thus, collections of song lyrics and their translations are not taken into account.

The following systems are examined: \textit{LyricsTraining}\footnote{\url{https://lyricstraining.com}}, \textit{FluentU}\footnote{\url{https://www.fluentu.com}}, \textit{LyricsGaps}\footnote{\url{https://www.lyricsgaps.com}}, and \textit{Language Zen}\footnote{\url{https://www.languagezen.com/en}}. All the four systems have a web version and a mobile version. Although the creators usually invest in one of the versions more, features of both web and mobile versions are discussed.

Different aspects of systems are analyzed: possible modes of discovering lyrics, type and structure of exercises for memorizing new words, language levels that the system offers, translation of separate words and whole lyrics into English. Moreover, the number of languages supported by systems and features not connected to song lyrics are mentioned. The systems that have more popular Android apps are highlighted first.

\section{LyricsTraining}

According to the Google Play\footnote{\url{https://play.google.com/store}}, an official app store for the Android operation system, the mobile version of \textit{LyricsTraining} is the most popular among the four systems mentioned here, and has more than 500,000 downloads\footnote{Accessed: June 28, 2019}. The system is solely developed for using music lyrics in language learning. When clicking on each song, the user can choose one of the two modes: karaoke mode and game mode. In game mode, a level should be selected: beginner, intermediate, advanced or expert. The more difficult the level is, the greater number of blanks in the text is. In web version, every song lacks either 10\%, 25\%, 50\% or 100\% of the words. In mobile version, number of words transformed into questions differs from song to song, but in expert level the user has to fill in all words in the lyrics, as in the web version. Such task is very unlikely to be accomplished, regardless of the language level of the user. Moreover, it hinders one of the main goals of learning vocabulary through song lyrics --- learning new words in context. Therefore, for expert level, it would be reasonable to decrease the number of words that are transformed into questions.

In all the other three levels in mobile version, the number of words extracted from the text seems to be dependent on the pace of the song. It is more comfortable than filling in 10\%, 25\% or 50\% of song lyrics, as in some very fast songs it might be too hard for the user to accomplish a higher level, which can possibly lead to his/her frustration.

Type of exercises in a mobile and in a web version are different: it is multiple choice questions and fill-in-the-blank questions accordingly. For every multiple choice question in the mobile version of \textit{LyricsTraining}, three distractors are available, all of them the words from the other blanks in the song. On the one hand, it makes it too easy for the user to guess the word while hearing it, as the distractors are not especially chosen to sound similarly to the right answer. On the other hand, the user is less stressed, as he/she needs less time to make a decision about the next word, because the three out of four options from the previous question stay the same in the next one. Apart from that, it is programmatically more simple to use words from other blanks as distractors, as there is no need in an external source for obtaining them. Moreover, due to the fact that the distractors need not be picked beforehand, this allows random generation of blank spaces, which is implemented in \textit{LyricsTraining}. 

Fill-in-the-blank questions in the web version are considerably harder than multiple choice questions, especially when the singer uses slang, and the song is rather fast. In a combination with choosing to fill 50\% of the lyrics, it might provoke user's dissatisfaction with his/her result. When comparing the game mode in the web version and in the mobile version, the latter appears to be more sensible and thought-out.

In the game mode, both in the web and in the mobile version, a music video is synchronized with the lyrics, and a certain time limit is implemented --- the user should choose the option or type in the answer as soon as the line was sung. If the user did not know the right word the first time, there is a possibility to hear the line again or to see the right word and go to the next blank space.

Apart from the game mode, the user can choose a karaoke mode. In this mode, as in the game mode, a music video is synchronized with the lyrics, so that the line sung at the moment is highlighted. Unlike in the game mode, however, user does not have to fill in any words and can switch the song backward and forward. This mode is suitable for either hearing the song for the first time, before the game mode, or afterwards, so that the user can memorize the new words he/she encountered.

Unlike some other systems, \textit{LyricsTraining} does not allow clicking on words in the song to obtain their translations and does not offer translations of the whole lyrics either. However, there is a section \textit{My vocabulary }in the mobile version, where separate words can be translated into 58 other languages through the Microsoft Translator.

Even though only the German version of the \textit{LyricsTraining} was tested, the system is offered in twelve other languages, among them e.g. Japanese and Turkish.

\section{FluentU}

The system with the second popular Android app is \textit{FluentU}, which has more than 100,000 downloads\footnote{Accessed: June 28, 2019}, according to Google Play. Music videos is only one of the resources for language learning the system offers. For the first two levels, Beginner 1 and Beginner 2, audios on grammar topics are available. For all the levels, videos on different topics (e.g. \textit{Arts and Entertainment} and \textit{Science and Tech}) are accessible. The videos have different formats, too, such as \textit{Mini-movies} or \textit{Talks/Speeches} --- and among them, \textit{Music videos}.

There are three possible ways one can work with a music video. First, there is a dialogue mode, where the user can read through song lyrics and its line-by-line translation into English. The vocabulary mode contains the words that are taken from the song text and are supposed to be learnt by the user through the quiz. The last mode is playing the video. In this mode, lyrics and videos are synchronized, and the user can go backward and forward. As \textit{FluentU} is one of the systems that allow translating separate words and phrases in the song text, the user can click on any word of the line that is being sung at the moment and acquire the word's translation, while the video is automatically paused. The translations are very detailed, they include the part of speech, some specific characteristics for this part of speech, such as gender for nouns, an audio with a native speaker pronouncing the word and general example sentences. In the web version, the translations are supplemented with example sentences from the song and a suitable picture. 

In order to train the vocabulary from the song, the user can do a quiz. There are three types of the exercises: multiple choice questions, fill-in-the-blank questions and ordering. Multiple choice questions include, for example, choosing a right translation of a sentence from English into German. An example of a fill-in-the blank questions would be selecting the right word to complete the sentence. Finally, ordering is concerned with putting the words int the right order to get a line from the song. The quiz is done after watching the video, and there is no time limit for answering the questions.

All the music videos are assigned levels. There are six levels in total: Beginner 1 and 2, Intermediate 1 and 2, Advanced 1 and 2. The more difficult the level, the less music videos is available --- thus, Advanced 1 and Advanced 2 have only two and one music videos accordingly. There are two ways of solving this problem: either more videos can be added or each music video could be adapted for other levels. There are forty videos in total, and there are some learner-specific songs among them as well --- for instance, \textit{The Umlaut Song!} or \textit{Numbers 1-10 Singalong}. This kind of songs are probably not that interesting for an average music lover, even though they might be helpful in memorizing the vocabulary.

Apart from German, nine more languages are implemented, including, for example, Korean and Russian.

Compared to \textit{LyricsTraining}, \textit{FluentU} uses a more complicated quiz-format, which can be more helpful for memorizing the vocabulary. However, \textit{LyricsTraining} offers interactive learning, which \textit{FluentU} lacks. This way of learning might be more entertaining for the user. A very small number of music videos is available in \textit{FluentU}, but there are some other activities for learning implemented in the system, so it cannot be considered a big disadvantage.  
The big advantage of \textit{FluentU}, on the other hand, is the possibility to translate words and phrases while listening to the song, as well as an access to translations of the whole lyrics, which is not found in \textit{LyricsTraining}.

\section{Language Zen}

The third most popular Android app belongs to \textit{Language Zen} and has more than 10,000 downloads\footnote{Accessed: June 28, 2019}, according to Google Play. As \textit{FluentU}, \textit{Language Zen} offers other ways to learn languages: the user can choose the courses mode, which contains phrases and vocabulary split into topics, or start learning the most frequent words. 

In contrast to other systems, \textit{FluentU} only offers two languages to learn. In web version, it is only Spanish for English speakers, and in the mobile version, there is also a possibility to learn English for Portuguese speakers. The creators of the system are planning to implement more languages in the future. Only the course in Spanish for English speakers was analyzed. 

When the user chooses a song, he/she can either select the play mode or the learn mode. In the play mode, lyrics and audios are synchronized, and the user can play the song forward and backward, as in the other systems discussed so far. \textit{Language Zen} also offers translations of individual words when clicking on them. Apart from that, every line is followed by its translation. Translations of some individual words (e.g. \textit{se}) are complemented with short grammar notes that explain their usage (e.g. \textit{Passive ``se''}). In the web version, word forms are also morphologically analyzed --- for example, a verb is assigned number, person and tense in present tense, and a noun --- number and gender.

The learn mode offers a quiz with multiple choice and fill-in-the-blank questions. The user is asked to translate words and phrases from Spanish to English and vice versa. An interesting feature of the system is the way things that the user learnt or attempted to learn before are included in the quiz even if a song which is new for the user is chosen. There is no time limit for answering questions.

There are many levels of Spanish the user can choose from: beginner, beginner plus, intermediate, intermediate plus, advanced, advanced plus and fluent. There is also a near native level, which is only present in the web version. However, to take any level higher than beginner, the user should pass an assessment test. Beginner level offers only fourteen songs; other levels were not examined. 

\section{LyricsGaps}

The system that has the least popular Android app among the four systems examined in this section is \textit{LyricsGaps}. According to Google Play, there are more than 500 downloads\footnote{Accessed: June 28, 2019} of this app. It is nevertheless important to point out that the web version of the system is much more developed than its mobile version. 

As \textit{LyricsTraining}, \textit{LyricsGaps} is fully devoted to learning languages through song lyrics. Exercise mode is shared by both the mobile and the web version. The web version, however, also offers a karaoke mode and a word-to-word mode. The karaoke mode displays the whole text, and, unlike in other systems, the song and its lyrics are not synchronized, so the line sung at the moment is not highlighted. The word-to-word mode is rather strange: it is supposed that the user has to click on every word of the song while playing the video.

In the web version, exercise  mode includes texts with multiple choice and fill-in-the-blank gaps. The type of questions depends on the level chosen by the user. Multiple choice questions offer too many distractors --- usually, all the words from other blank spaces are used for that. Using only a few distractors, as in \textit{LyricsTraining}, would be more benefiting for users in this case, as it will not confuse them that much. In the mobile version, only three distractors are used. However, every line has a blank space that should be filled with one of the four possible options, which results in more gaps than in the web version.

In some sense, one can speak about time limit in the mobile version, as the user has to type in an answer fast enough to follow the song. If he/she does not type the answer in time, the song does not stop, as, for instance, in \textit{LyricsTraining}. This kind of design might have a negative impact on the results of the user and their satisfaction with the system. In the web version, there is no time limit, and the user can listen to the song as many times as he/she would like.

As has been mentioned before, in the web version the user has the opportunity to choose a level of the exercise: beginner, intermediate or expert. Beginner level exercises include multiple choice questions, and the other two levels offer questions of fill-in-the-blank type, which are objectively harder for the user. No correlation between the level and the number of gaps in the text is found, although there are on average more gaps in songs with longer texts\footnote{Five songs from different artists were analyzed}. 

Even though the system does not provide translations of whole lyrics, the user can acquire translations of separate words by clicking on them. If the word form is different from its lemma, the translation of the word form will be shown --- for example, \textit{gab} would be translated into English as \textit{gave}. Apart from English, translations into nineteen other languages are possible. Unfortunately, translation of words is only available in the web version of the system.

\textit{LyricsGaps} is accustomed to twenty languages, not including German. Compared to three other systems, it is the greatest number of languages offered. Among popular languages, such as French and Chinese, more exotic ones are offered --- for instance, Indonesian or Catalan.

To sum up, there are two systems that offer exercises while listening to a song, and two systems where exercises should be done after listening. A system can either contain different songs for different levels or adapt each song to several levels. In most of the systems, songs are synchronized with their lyrics, and the user is able to click backward and forward buttons to control the song. The maximum number of languages offered is twenty-one, and the least is two. In three out of four systems, translations of separate words are possible. Two of these systems also contain translations of whole lyrics.
